% Carta de intención para el IRC
\setuppapersize[letter]
\setuppagenumbering[location=footer]
\setupwhitespace[big]
\setuplayout[textwidth=13.5cm]

\starttext
\rightaligned{{\em Caracas, 22 de mayo de 2020.}}

\subject{Estimados miembros del International Rescue Commitee de Venezuela:}

Es un placer para mí el poder presentar ante ustedes mi historia de vida y mis habilidades. Durante los últimos cinco años he trabajado arduamente especializandome en el área de la educación, y la salud sexual y reproductiva.
Con agrado y amor he alcanzado metas hermosas, como la promoción de la educación integral en sexualidad entre más de veinte mil jóvenes y adultos, cerca de seis mil al año.
Adolescentes, estudiantes universitarios, docentes, personal médico y padres, se han beneficiado de las iniciativas que con esfuerzo he planificado, organizado, facilitado y supervisado en ese periodo de tiempo.

También, he sido responsable de brindar acceso a salud sexual y reproductiva, con métodos anticonceptivos, consultas prenatales, ginecología y ecología a más de sesenta mil mujeres beneficiadas, como implementador y coordinador de diversos proyectos.

A su vez, he sido reconocido internacionalmente como experto en el área de educación integral en sexualidad, género, servicios amigables para adolescentes y jóvenes, participación juvenil, y salud sexual y reproductiva.

No es una casualidad, pues desde mi formación universitaria he sido motivado por los valores de igualdad de derechos humanos, combate a la discriminación y la búsqueda de cambios sociales positivos.
Quizás debido a mi pertenencia a la comunidad LGBTIQ+, mi interés siempre ha sido dirigido hacia la promoción de la justicia social en el ámbito de la sexualidad.
Esto me llevó a la elaboración de un trabajo de grado dirigido a explorar y reivindicar las experiencias de las personas trans.
Y más tarde, a buscar una línea laboral profesional acorde con mis principios e intereses.

Por eso, al recibir noticias de la vacante para un cargo en su organización, me siento llamado a continuar en la búsqueda de expandir mi vocación profesional.
Aunque mi trabajo actual como especialista de programas en PLAFAM, A.C. me ha brindado la oportunidad de aprender y ejercitar mis habilidades profesionales, a su vez ha agotado el espacio para la expansión de mis horizontes y aspiraciones.
Es de gran interés para mí ubicar un espacio profesional donde pueda continuar con el aprendizaje y ejercicio de habilidades en el espacio gerencial.
En los últimos años he gerenciado, como especialista, la formación y las intervenciones de cientos de educadores, pasantes, psicólogos y facilitadores, así como de más de doscientos voluntarios jóvenes, en la ejecución de múltiples proyectos educativos y de salud.

Espero entonces, ser considerado por ustedes para el cargo de Gerente de Salud Sexual y Reproductiva, y poder desde esa posición hacer un uso más completo de mi potencial y mis conocimientos
Es mi deseo seguir asumiendo nuevos retos y oportunidades de crecimiento, continuando, a su vez, mi vocación por la promoción de la salud, la igualdad de derechos, la participación democrática y el acceso a la educación.
Siento que dentro de su organización, podré no sólo cumplir con estas aspiraciones personales, sino que también podré aportar al cumplimiento efectivo de sus objetivos y su misión.

Estamos en medio de una crisis de salud producto de la pandemia global causada por el SARS-CoV-2, pero también es cierto que las crisis son una ventana de oportunidad para promover cambios positivos.
Esta oportunidad puede ser aprovechada para continuar brindando el apoyo y asistenacia humanitaria que requieren mis hermanos Venezolanos refugiados en búsqueda de su supervivencia.

Saludos y espero pronto escuchar noticias de su parte.

\blank[6*big]
\startlinecorrection
\midaligned{Leonardo Pérez \\ {\em Licenciado en Psicología Social} \\ leo.prez.k@gmail.com \\ (+58 416) 427 2880}
\stoplinecorrection
\stoptext
