% Cover letter para postulación a Save the Children

\documentclass[draft=false]{scrlttr2}
\usepackage{komacv-lco, fontspec, amsmath, microtype}
\usepackage{polyglossia}
\setdefaultlanguage{spanish}
\KOMAoptions{fromemail=true, fromextrainfo=true, fromlinkedin=true, frommobilephone=true, fromphone=true, foldmarks=false}
\LoadLetterOption{leonardo}
\setmainfont{Gentium Basic}

\begin{document}
    \begin{letter}{Save the Children International}
    \setkomavar{subject}{Postulación al cargo de Gerente de Proyectos Humanitarios}

\opening{Estimada/os,}

Es para mí un placer presentar ante ustedes mi historia de vida y mis habilidades.
Con la esperanza de obtener su consideración al cargo de gerente de proyectos humanitarios.
Durante los últimos cinco años he trabajado arduamente especializándome en la gestión de proyectos sociales, poniendo un énfasis en el área de la educación, y la salud sexual y reproductiva.
Con agrado y amor he alcanzado importantes metas, como la promoción de la educación integral en sexualidad entre más de veinte mil jóvenes y adultos, cerca de seis mil al año.
Adolescentes, estudiantes universitarios, docentes, personal médico y padres, se han beneficiado de las iniciativas que he planificado, organizado, facilitado y supervisado en ese periodo de tiempo.
También, he sido responsable de brindar acceso a salud sexual y reproductiva, con métodos anticonceptivos, consultas prenatales, ginecología y ecología a más de sesenta mil mujeres beneficiadas, como implementador y coordinador de diversos proyectos a nivel nacional.

A su vez, he sido reconocido internacionalmente como experto en el área de educación integral en sexualidad, género, servicios amigables para adolescentes y jóvenes, participación juvenil, y salud sexual y reproductiva.

No es una casualidad, pues desde mi formación universitaria he sido motivado por los valores de igualdad de derechos humanos, combate a la discriminación y la búsqueda de cambios sociales positivos.
Quizás debido a mi pertenencia a la comunidad LGBTIQ+, mi interés siempre ha sido dirigido hacia la promoción de la justicia social.
Esto me llevó a la elaboración de un trabajo de grado dirigido a explorar y reivindicar las experiencias de las personas trans.
Y más tarde, a buscar una línea laboral profesional acorde con mis principios e intereses.

Por eso, al recibir noticias de la vacante para un cargo en su organización, me siento llamado a continuar en la búsqueda de expandir mi vocación profesional.
Aunque mi trabajo actual como especialista de programas en Plafam, A.C. me ha brindado la oportunidad de aprender y ejercitar mis habilidades profesionales, a su vez ha agotado el espacio para nuevas experiencias y desarrollo profesional.
En los últimos años he gerenciado, como especialista, la formación y las intervenciones de cientos de educadores, pasantes, psicólogos y facilitadores, así como de más de doscientos voluntarios jóvenes, en la ejecución de múltiples proyectos educativos y de salud.
Y es mi aspiración continuar la educación formal en el área de gestión de proyectos sociales con certificación y estudios de maestría.

Además, he sido el representante de mi actual organización frente al grupo de apoyo humanitario organizado por OCHA en Venezuela desde hace dos años.
Allí he participado en las reuniones de las mesas de salud y protección de niños, niñas y adolescentes.
Por lo cual, no soy extraño ante el sistema de apoyo humanitario y el trabajo con agencias internacionales.

Espero entonces, ser considerado por ustedes para el cargo de Gerente de Proyectos Humanitarios, y poder desde esa posición hacer un uso más completo de mi potencial y mis conocimientos.
Es mi deseo seguir asumiendo nuevos retos y oportunidades de crecimiento, aportando con mis conocimientos y trabajo en la protección de los niños y niñas de Venezuela.
Siento que dentro de su organización, podré no sólo cumplir con estas aspiraciones personales, sino que también podré aportar al cumplimiento efectivo de sus objetivos y su misión.

Estamos en medio de una crisis de salud producto de la pandemia global causada por el SARS\nobreakdash-CoV\nobreakdash-2, pero también es cierto que las crisis son una ventana de oportunidad para promover cambios positivos.
Esta oportunidad puede ser aprovechada para continuar brindando el apoyo y asistencia humanitaria que requieren los niños y niñas venezolanos.

% El titular del puesto será responsable de la operatividad del proyecto "Integrated MPC 'Plus' Assistance to Vulnerable and Crisis-Affected Venezuelans" en marco del programa de respuesta humanitaria de SC en Venezuela. Debe garantizar que se ejecute las actividades y presupuesto de manera oportuna y con calidad. Esta posición estará a cargo de la gestión del proyecto a nivel nacional de manera integral y se coordinará de manera permanente con los especialistas nacionales, el equipo de trabajo del proyecto y los socios locales.
%
%
%
% Esta posición estará reportando al Deputy Team Leader, en la Oficina de Respuesta Interna para Venezuela.
%
% Proporcionará liderazgo estratégico y operativo para desarrollar e implementar un programa exitoso. Garantizará el cumplimiento de la calidad técnica y programática general en la implementación, el cumplimiento de las normas, regulaciones de donantes y la presentación oportuna de todos los entregables, incluidos los planes de trabajo anuales, los planes de monitoreo del desempeño, los informes según sea necesario. Es responsable de la dirección general y la coordinación de las actividades de cualquier socio sub receptor bajo el acuerdo en coordinación con el equipo gerencial de SC Venezuela.
%
%
%
% Para tener éxito traerás / tendrás:
%
%     Educación Universitaria en Ciencias Humanas, Sociales, económicas, Políticas, Derecho o afines.
%     Especialización o postgrado deseable en temas relacionados con seguridad alimentaria, administración pública, gestión de proyecto.
%     Excelente manejo de paquetes informáticos: Word, Excel, Power Point, Outlook, etc.
%     Fluidez en español, escrito y hablado
%     Fluidez en Inglés, escrito y hablado
\closing{Les envío saludos a todos los miembros de su organización y espero pronto escuchar noticias de su parte.}
    \end{letter}
\end{document}
