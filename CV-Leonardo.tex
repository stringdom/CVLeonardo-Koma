% Implementación de un CV en el formato de KOMA-description

% Fecha de inicio y salida.
% Nombre de la compañía.
% Cargo
% Descripción de cargo. (Al menos 5 bullets)
%
% Estudios
%
% Softwares

\documentclass[xcolor=svgnames,color=DarkSlateGray]{komacv}
% \usepackage{fontawesome5}

\renewcommand*\title{CV Leonardo Pérez 2020}
\renewcommand*\acadtitle{Licenciado en Psicología Social}
\renewcommand*\firstname{Leonardo Rafael\\}
\renewcommand*\familyname{Pérez Martínez}
\renewcommand*\mobile{+58 416 427 28 80}
\renewcommand*\phonenr{+58 212 472 79 82}
\renewcommand*\extrainfo{\faSkype \link[Stringdom42]{https://join.skype.com/invite/xzSFqvGk6nBK}}
\renewcommand*\email{leo.prez.k@gmail.com}
\renewcommand*\linkedin{\link[psic-leonardo-pérez]{www.linkedin.com/in/psic-leonardo-pérez}}
\renewcommand*\addressstreet{Av.~Intercomunal de Antímano con Calle~El~Carmen}
\renewcommand*\addresscity{Caracas, Municipio Libertador}
\photo[mframe]{0.2\textwidth}{foto.jpg}

\pagestyle{scrheadings}
\clearscrheadfoot
\ofoot{\pagemark/\totalpagemark}

\begin{document}
\maketitle

\section{Información profesional}
\subsection{Perfil}
\raggedright
    Psicólogo social con conocimiento avanzado de informática. Diseñador y coordinador de proyectos de incidencia social. Redactor y editor de textos técnicos, manuales, y contenido instruccional y promocional para redes sociales. Orador experimentado. Facilitador de grupos y talleres con conocimiento en evaluación basada en competencias. Especialista en educación, derechos, y salud sexual y reproductiva.
% \subsection{Habilidades}
% \raggedright
%     Experiencia en diseño e implementación de proyectos. Planificacion, diseño curricular, monitoreo de indicadores de desempeño, control y ejecución de presupuestos, reporte financiero y narrativo de resultados. Manejo, formación y facilitación de grupos. Capacidad para gestión, organización, supervisión, y manejo de personal y equipos de trabajo, así como de grupos de advocacy.

\section{Experiencia Laboral}
    % \subsection{Empleos}
        \cventry{Nov, 2019-Presente}{CEX/CIES}{Miembro del comité técnico del centro de excelencia latinoamericano en educación integral en sexualidad}{}{}{}
        \cvlistitem{Validación de los contenidos educativos del curso online de Educación Integral en Sexualidad para prestadores de servicios de planificación familiar.}
        \cvlistitem{Creación y selección de una caja de herramientas para la EIS.}
        \cvlistitem{Coordinar la implementación del proyecto de participación y liderazgo juvenil.}
        \cvlistitem{Realizar consultoría a los miembros del centro de excelencia en temas de salud y derechos sexuales y reproductivos.}
        \cvlistitem{Participar en las reuniones remotas del comité técnico.}

        \newpage

        \cventry{Sept, 2017-Actual}{PLAFAM,~A.C}{Especialista del programa de jóvenes y adolescentes}{}{}{}
        \cvlistitem{Ejecución de proyectos sociales para la participación de jóvenes y adolescentes en iniciativas de promoción de la EIS, planificación familiar y empoderamiento en los derechos sexuales y reproductivos.}
        \cvlistitem{Dirección y coordinación del grupo de voluntariado Plafam Juvenil.}
        \cvlistitem{Promover y coordinar la participación de los jóvenes dentro y fuera de la institución.}
        \cvlistitem{Realizar alianzas con instituciones hermanas, centros educativos, grupos juveniles, instituciones comunitarias públicas y privadas.}
        \cvlistitem{Coordinar y ejecutar actividades formativas para personas con incidencia en jóvenes, como docentes, representantes, médicos, enfermeras, líderes juveniles y otros adultos significativos.}

        \cventry{Feb-Ago, 2017}{teleSUR, La nueva televisión del sur,~C.A}{Redactor Web}{}{}{}
        \cvlistitem{Redaccion de artículos de investigación sobre eventos históricos.}
        \cvlistitem{Redacción de contenidos para la página web del canal.}
        \cvlistitem{Resumen de acontecimientos noticiosos.}
        \cvlistitem{Elaboración de notas informativas sobre entrevistas.}
        \cvlistitem{Creación de notas promocionales para la programación del canal.}

        \cventry{2015-2016}{Kavanayentech~C.A.}{Consultor de tecnología en implementación de APIA~Documentum}{}{}{}
        \cvlistitem{Consultoría psicológica en adopcion de tecnologías de la información en la administración pública.}
        \cvlistitem{Consultoría y supervisión del equipo de implementación del paquete informático APIA~Documentum.}
        \cvlistitem{Elaboración y ejecución de talleres de entrenamiento al personal en el uso de tecnologías para trámites administrativos.}
        \cvlistitem{Elaboración y ejecución de un plan de adopción de la tecnología APIA~Documentum para el personal de administración pública.}
        \cvlistitem{Creación y aplicación de instrumentos de evaluación de la adopción de nuevas tecnologías.}

        \cventry{2010-2015}{IBM de Venezuela~C.A.}{Operador de ‘Centro de innovación de negocios’}{}{}{}
        \cvlistitem{Monitoreo de los procesos informáticos de la plataforma interbancaria nacional.}
        \cvlistitem{Registro y seguimiento de incidencias y eventos de la plataforma.}
        \cvlistitem{Ejecución de procesos informáticos de rutina y mantenimiento.}
        \cvlistitem{Ejecución de tareas de rutina en los equipos informáticos del datacenter.}
        \cvlistitem{Asistencia al personal desarrollador de software y al equipo técnico de mantenimiento del datacenter.}

        \cventry{2010-2011}{The Adecco Group Venezuela—Banco Mercantil}{Analista de help-desk}{}{}{}
        \cvlistitem{Soporte técnico informático para el personal administrativo del Banco Mercantil.}
        \cvlistitem{Recepcion de llamadas de soporte.}
        \cvlistitem{Seguimiento a eventos de soporte técnico.}
        \cvlistitem{Asistencia en la ejecución de procesos informáticos administrativos.}
        \cvlistitem{Soporte remoto de incidencias informáticas.}

\section{Educación}
    \cventry{2018}{Psicología mención Psicología Social}{Universidad Central de Venezuela}{}{Licenciado}{}
    \cventry{2005}{Bachiller en Ciencias}{U.E.N. Rafael Seijas}{}{}{}

    \subsection{Software}
        \cvitem{Microsoft Office}{Manejo avanzado de Word, Excel, PowerPoint, y Project. Conocimientos básicos de Access.}
        \cvitem{RStudio}{Análisis de datos estadísticos básicos con el paquete Tidyverse en el lenguaje de análisis R.}
        \cvitem{Git}{Gestión avanzada de control de versiones y trabajo colaborativo en la nube.}
        \cvitem{\LaTeX Typesetting software}{Elaboración avanzada y automatizada de documentos de alta calidad tipográfica.}

    \subsection{Idiomas}
        \cvitem{Español}{Lengua materna.}
        \cvitem{Inglés}{Fluidez en lectura, escritura y conversación.}

\newpage

\section{Coordinación de proyectos}
    \cvitem{2020}{Esperanza: Programa de formación de prestadores de servicios de salud en derechos y salud, sexual y reproductiva.}
    \cvitem{2018-2020}{Desarrollo de un modelo integral para la prevención y atención del sindrome por virus Zika en Venezuela: un enforque desde la salud sexual y reproductiva.}
    \cvitem{2017-2018}{Háblalo: Visibilización y abordaje del estigma contra el aborto en Venezuela mediante mensajes adiovisuales.}
    \cvitem{2016-2017}{Cero prejuicios, PLAFAM habla contigo: avance de la educación integral en sexualidad en liceos de Caracas, Venezuela.}
    \cvitem{2015-2016}{APIA~Documentum: Prueba piloto para la digitalización de procesos de legalización de documentos con énfasis en ‘cero papel’ en el Registro Principal de Caracas del SAREN.}

\section{Referencias laborales}
        \begin{compactdesc}
                \item{Nelmary Diaz — }{\em Gerente de programas, PLAFAM, A.C.}
                \item{\bfseries Email:} \emaillink{ndiaz@plafam.org.ve}
                \item{\bfseries Teléfono:} +58 412 954 9116 / +58 212 634 0177
        \end{compactdesc}

\medskip

        \begin{compactdesc}
                \item{Gustavo Torres — }{\em Fundador y director ejecutivo, Kavanayentech, C.A.}
                \item{\bfseries Email:} \emaillink{gtorres@kavanayentech.com}
                \item{\bfseries Teléfono:} +1 786 487 7791
        \end{compactdesc}
\end{document}
