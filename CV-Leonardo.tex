% Implementación de un CV en el formato de KOMA-description
% 18/12/2018 Primera versión

\documentclass[xcolor=svgnames,color=DarkSlateGray]{komacv}

\renewcommand*\title{CV Leonardo Pérez 2020}
\renewcommand*\acadtitle{Licenciado en Psicología Social}
\renewcommand*\firstname{Leonardo Rafael\\}
\renewcommand*\familyname{Pérez Martínez}
\renewcommand*\mobile{(0416) 427 28 80}
\renewcommand*\phonenr{(0212) 472 79 82}
\renewcommand*\email{leo.prez.k@gmail.com}
\renewcommand*\addressstreet{Av.~Intercomunal de Antímano con Calle~El~Carmen}
\renewcommand*\addresscity{Caracas, Municipio Libertador}
\photo[mframe]{0.2\textwidth}{foto.jpg}

\begin{document}
\maketitle

\section{Información profesional}
\subsection{Perfil}
\raggedright
  Psicólogo social con conocimiento avanzado de informática. Diseñador y coordinador de proyectos de incidencia social. Redactor y editor de textos y contenido. Orador experimentado. Facilitador de grupos y talleres.
\subsection{Habilidades}
\raggedright
  Experiencia en diseño e implementación de proyectos. Planificacion, diseño curricular, monitoreo de indicadores de desempeño, control y ejecución de presupuestos, reporte financiero y narrativo de resultados. Manejo, formación y facilitación de grupos. Capacidad para gestión, supervisión y manejo de personal y equipos de trabajo.

\section{Educación}
    \cventry{2018}{Psicología mención Psicología Social}{UCV}{}{Licenciado}{}
    \cventry{2005}{Bachiller en Ciencias}{U.E.N. Rafael Seijas}{}{}{}

\subsection{Idiomas}
\cvitem{Español}{Lengua materna.}
\cvitem{Inglés}{Fluidez en lectura, escritura y conversación.}

\section{Experiencia Laboral}
\subsection{Coordinación de proyectos}
  \cvitem{2018-2020}{Desarrollo de un modelo integral para la prevención y atención del sindrome por virus Zika en Venezuela: un enforque desde la salud sexual y reproductiva.}
  \cvitem{2017-2018}{Háblalo: Visibilización y abordaje del estigma contra el aborto en Venezuela mediante mensajes adiovisuales.}
  \cvitem{2017}{Cero prejuicios, PLAFAM habla contigo: avance de la educación integral en sexualidad en liceos de Venezuela.}

\subsection{Empleos}
  \cventry{2017-2020}{Especialista del programa de jóvenes y adolescentes}{PLAFAM,~A.C}{}{}{}
  \cventry{2017}{Redactor Web}{teleSUR, La nueva televisión del sur,~C.A}{}{}{}
  \cventry{2015-2016}{Consultor de tecnología en implementación de APIA~Documentum}{Kavanayentech}{}{}{}
  \cventry{2013-2015}{Operador de ‘Centro de innovación de negocios’}{IBM de Venezuela}{}{}{}
  \cventry{2010-2013}{Operador de ‘Centro de innovación de negocios’}{e-Power Outsourcing,~S.A.—IBM}{}{}{}
  \cventry{2010-2011}{Analista de help-desk}{The Adecco Group Venezuela—Banco Mercantil}{}{}{}
  \cventry{2006-2007}{Analista de resultados}{Emevenca Venezuela}{}{}{}

\section{Referencias laborales}
    \begin{compactdesc}
        \item{Nelmary Diaz — }{\em Gerente de programas, PLAFAM, A.C.}
        \item{\bfseries Email:} \emaillink{ndiaz@plafam.org.ve}
        \item{\bfseries Teléfono:} 0412 954 9116 / 0212 634 0177
    \end{compactdesc}

\medskip

    \begin{compactdesc}
        \item{Gustavo Torres — }{\em Fundador y director ejecutivo, Kavanayentech, C.A.}
        \item{\bfseries Email:} \emaillink{gtorres@kavanayentech.com}
        \item{\bfseries Teléfono:} +1 786 487 7791
    \end{compactdesc}
\end{document}
