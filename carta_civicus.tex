% Carta de intención para el IRC
\setuppapersize[letter]
\setupwhitespace[none]
\setupindenting[yes, medium]
\setuplayout[textwidth=15cm]
\setupheader[state=normal,style=small]
\setupbodyfont[palatino,11pt]
\usesymbols[fontawesome]

\startsetups[headertext]
    \startframed[frame=off,align=flushleft,width=fit]
    Leonardo Rafael Pérez Martínez.\\
    Caracas, Venezuela.\\
    \symbol[fontawesome][phone] +58 416 427 2880\\
    \symbol[fontawesome][envelope] leo.prez.k@gmail.com
    \stopframed
\stopsetups

\setupheadertexts[\directsetup{headertext}][]


\starttext
\setuppagenumber[state=stop]
\blank[big]
CIVICUS

\rightaligned{\currentdate}

\subject{{\em Application for the position of “Youth Digital Engagement Officer”.}}

To Whom It May Concern:

From my time in college, studying psychology, I learned that communication is the defining quality of humans. Speaking more than one language, for example, brings the possibilities for social connection to millions of people. Communicating with others releases the collaborative power of humanity.

Now that I have worked with so many groups and communities professionally, this belief on communication has been ratified several times over. To create alliances, build communities, and connect activists with each other, takes a two way channel of trust and learning. To me it is fulfilling to observe young activists and colleagues realize their potential, while at the same time learning and growing through the work we do together. This has been accompanied with the satisfaction of growing a career from teaching and implementing advocacy engagement events to a position of leadership and management. This growth has allowed me to pursue a passion for working with interdisciplinary teams that feed my curiosity, continious desire to learn and to share this lessons with everyone who is willing to listen.

An opportunity to work with CIVICUS means I can pursue this professional passion, and further take advantage of the power of bilingüal communication. In order to empower people of all ages and regions of the world to exercise their rights, while creating a more democratic and peaceful world. It takes effort to organize and design projects that take all individual's well-being into account. I believe that I can contribute with this endeavor.

Now that we live in a world where we deal with the biggest pandemic in living memory, I've strenghted the value I place on remote work and the power of tecnology to connect us. I've been reassured on my hability to overcome the digital hurdles and take full advantage of the perks of working from home. As part of the first digital native generation, I've come to seamlessly integrate my workflow with the myriad of software tools available, becoming familiar with the most common, as well as the most obscure. Recognizing their uses and niche strenghts and weaknesses, to bring about the highest quality results in every circumstance.

All together, it is my intention to take all opportunities to grow, improve and also change the world we share for the better. I hope that this can be made possible by collaborating with you. Thanks for your consideration.

Best regards.

\whitespace
\midaligned{Leonardo Pérez}
\stoptext
